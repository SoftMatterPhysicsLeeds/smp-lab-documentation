%% Generated by Sphinx.
\def\sphinxdocclass{report}
\documentclass[letterpaper,10pt,english]{sphinxmanual}
\ifdefined\pdfpxdimen
   \let\sphinxpxdimen\pdfpxdimen\else\newdimen\sphinxpxdimen
\fi \sphinxpxdimen=.75bp\relax
\ifdefined\pdfimageresolution
    \pdfimageresolution= \numexpr \dimexpr1in\relax/\sphinxpxdimen\relax
\fi
%% let collapsible pdf bookmarks panel have high depth per default
\PassOptionsToPackage{bookmarksdepth=5}{hyperref}

\PassOptionsToPackage{warn}{textcomp}
\usepackage[utf8]{inputenc}
\ifdefined\DeclareUnicodeCharacter
% support both utf8 and utf8x syntaxes
  \ifdefined\DeclareUnicodeCharacterAsOptional
    \def\sphinxDUC#1{\DeclareUnicodeCharacter{"#1}}
  \else
    \let\sphinxDUC\DeclareUnicodeCharacter
  \fi
  \sphinxDUC{00A0}{\nobreakspace}
  \sphinxDUC{2500}{\sphinxunichar{2500}}
  \sphinxDUC{2502}{\sphinxunichar{2502}}
  \sphinxDUC{2514}{\sphinxunichar{2514}}
  \sphinxDUC{251C}{\sphinxunichar{251C}}
  \sphinxDUC{2572}{\textbackslash}
\fi
\usepackage{cmap}
\usepackage[T1]{fontenc}
\usepackage{amsmath,amssymb,amstext}
\usepackage{babel}



\usepackage{tgtermes}
\usepackage{tgheros}
\renewcommand{\ttdefault}{txtt}



\usepackage[Bjarne]{fncychap}
\usepackage{sphinx}

\fvset{fontsize=auto}
\usepackage{geometry}


% Include hyperref last.
\usepackage{hyperref}
% Fix anchor placement for figures with captions.
\usepackage{hypcap}% it must be loaded after hyperref.
% Set up styles of URL: it should be placed after hyperref.
\urlstyle{same}

\addto\captionsenglish{\renewcommand{\contentsname}{Contents:}}

\usepackage{sphinxmessages}
\setcounter{tocdepth}{1}



\title{SMP Labs}
\date{Jul 26, 2022}
\release{1}
\author{Daniel Baker}
\newcommand{\sphinxlogo}{\vbox{}}
\renewcommand{\releasename}{Release}
\makeindex
\begin{document}

\pagestyle{empty}
\sphinxmaketitle
\pagestyle{plain}
\sphinxtableofcontents
\pagestyle{normal}
\phantomsection\label{\detokenize{index::doc}}



\chapter{Anton\sphinxhyphen{}Paar Rheometer}
\label{\detokenize{anton-paar:anton-paar-rheometer}}\label{\detokenize{anton-paar::doc}}

\section{Initial Set\sphinxhyphen{}Up}
\label{\detokenize{anton-paar:initial-set-up}}\begin{enumerate}
\sphinxsetlistlabels{\arabic}{enumi}{enumii}{}{.}%
\item {} 
\sphinxAtStartPar
Open compressed air regulator (should be set to 100PSI).

\item {} 
\sphinxAtStartPar
Turn on the water bath.

\item {} 
\sphinxAtStartPar
Turn on the rheometer.

\item {} 
\sphinxAtStartPar
Turn on the PC and load Rheocompass (control software)

\item {} 
\sphinxAtStartPar
Attach the relevant base plate. Tools for doing so are located next to the machine.

\item {} 
\sphinxAtStartPar
Open the ‘Control Panel’ (right hand side of the screen) and click on ‘Initialise’.

\end{enumerate}

\noindent\sphinxincludegraphics{{_static/anton-paar/control-panel}.png}
\begin{enumerate}
\sphinxsetlistlabels{\arabic}{enumi}{enumii}{}{.}%
\setcounter{enumi}{6}
\item {} 
\sphinxAtStartPar
Attach measuring tool to rheometer, set temperature and zero the gap using the same panel used for initialisation.

\end{enumerate}

\noindent\sphinxincludegraphics{{_static/anton-paar/zero-gap}.png}


\section{Calibration}
\label{\detokenize{anton-paar:calibration}}
\begin{sphinxadmonition}{note}{Note:}
\sphinxAtStartPar
The calibration procedure should be performed at the beginning of your set of measurements using the sample geometry you intend to measure with.
\end{sphinxadmonition}
\begin{enumerate}
\sphinxsetlistlabels{\arabic}{enumi}{enumii}{}{.}%
\item {} 
\sphinxAtStartPar
Go to ‘My Apps’ and click on ‘Verification and Adjustment’ at the bottom of the page.

\end{enumerate}

\noindent\sphinxincludegraphics{{_static/anton-paar/my-apps}.png}
\begin{enumerate}
\sphinxsetlistlabels{\arabic}{enumi}{enumii}{}{.}%
\setcounter{enumi}{1}
\item {} 
\sphinxAtStartPar
Click on ‘Adjust Drive’

\end{enumerate}

\noindent\sphinxincludegraphics{{_static/anton-paar/verification}.png}


\chapter{Stelar FFC NMR}
\label{\detokenize{stelar-ffc:stelar-ffc-nmr}}\label{\detokenize{stelar-ffc::doc}}

\section{Hardware Setup}
\label{\detokenize{stelar-ffc:hardware-setup}}\begin{enumerate}
\sphinxsetlistlabels{\arabic}{enumi}{enumii}{}{.}%
\item {} 
\sphinxAtStartPar
Turn on the Galden recirculator

\item {} 
\sphinxAtStartPar
If compressed air supply isn’t on already, open valve.

\item {} 
\sphinxAtStartPar
If the PC isn’t on already, press orange ‘Reset computer array’ button

\item {} 
\sphinxAtStartPar
Press green TX button (should light up and fan should start).

\item {} 
\sphinxAtStartPar
Sample Setup
\begin{enumerate}
\sphinxsetlistlabels{\alph}{enumii}{enumiii}{}{.}%
\item {} 
\sphinxAtStartPar
Add white ‘tube holder’ to sample tube and centre your sample using the block on the side of the machine.

\item {} 
\sphinxAtStartPar
Place sample into magnet (place into hole in the top)

\item {} 
\sphinxAtStartPar
Place weight over tube above white holder to stop the tube floating in the compressed air.

\end{enumerate}

\item {} \begin{description}
\item[{Switch on VTC 90 (temperature controller)}] \leavevmode\begin{enumerate}
\sphinxsetlistlabels{\alph}{enumii}{enumiii}{}{.}%
\item {} 
\sphinxAtStartPar
Press RESET then 1 then ENTER (within a couple of seconds).

\item {} 
\sphinxAtStartPar
The display should now read “Flow = 300”. Press enter (don’t change the flow)

\item {} 
\sphinxAtStartPar
Enter desired starting temperature.

\end{enumerate}

\end{description}

\end{enumerate}


\section{Software Setup}
\label{\detokenize{stelar-ffc:software-setup}}\begin{enumerate}
\sphinxsetlistlabels{\arabic}{enumi}{enumii}{}{.}%
\item {} 
\sphinxAtStartPar
Load “ACQNMR” software (link on Desktop)

\item {} 
\sphinxAtStartPar
Navigate to Hardware =\textgreater{} Magnet =\textgreater{} Switch On.

\item {} 
\sphinxAtStartPar
Right click “Not Connected” (top right) and click “Connect”. You should see the front panel of the VTC unit replicated in this box. You can now change the sample temperature by right clicking this box and selecting ‘Set Point’.

\end{enumerate}


\section{Measurement Setup: Correcting the frequecy offset, F1.}
\label{\detokenize{stelar-ffc:measurement-setup-correcting-the-frequecy-offset-f1}}
\sphinxAtStartPar
You should perform these steps every time you change sample/temperature.
\begin{enumerate}
\sphinxsetlistlabels{\arabic}{enumi}{enumii}{}{.}%
\item {} 
\sphinxAtStartPar
On the ‘Main Par’ tab, select ‘EXP’ and choose ‘NP’ (we are going to use this 1 pulse sequence to set frequency and phasing).

\item {} 
\sphinxAtStartPar
Relaxation Field 7MHz.

\item {} 
\sphinxAtStartPar
File, New File Name.

\item {} 
\sphinxAtStartPar
Recycle delay to 5T1 in the ‘Main par’ tab (if you don’t know T1, choose 1s).

\item {} 
\sphinxAtStartPar
If desired, change the number of scans in the ‘Acq. Par \textendash{} Basic’ tab. 1\sphinxhyphen{}4 is usually sufficient.

\item {} 
\sphinxAtStartPar
Delay tau to 5T1 (if you don’t know T1, choose 1s).

\item {} 
\sphinxAtStartPar
Click on the GUN to start the measurement

\item {} 
\sphinxAtStartPar
Click the Save icon.

\item {} 
\sphinxAtStartPar
Click Correct F1 button.

\end{enumerate}


\section{Measurement Setup: Measuring R1/T1 spectrum at 7MHz.}
\label{\detokenize{stelar-ffc:measurement-setup-measuring-r1-t1-spectrum-at-7mhz}}\begin{enumerate}
\sphinxsetlistlabels{\arabic}{enumi}{enumii}{}{.}%
\item {} 
\sphinxAtStartPar
Change experiment to “NP/S” on the Main tab.

\item {} 
\sphinxAtStartPar
Make sure Recycle Delay is still roughly 5T1 (or 1s if you don’t know)

\item {} 
\sphinxAtStartPar
Set Relaxation Field to 7 MHz (units are MHz)

\item {} 
\sphinxAtStartPar
Set Maximum T1 to what you think T1 is (if you don’t know, then assume 0.2s i.e. 1s/5).

\item {} 
\sphinxAtStartPar
Click GUN \sphinxhyphen{} wait for measurement to finish.

\item {} 
\sphinxAtStartPar
Click SAVE

\item {} 
\sphinxAtStartPar
Click Evaluation Dialog. Choose RAM from list on right hand side.

\item {} 
\sphinxAtStartPar
If the measured T1 is much longer (\textgreater{}120\%) than your initial guess of T1Max then repeat this section. Similarly, if T1 is much shorter than you thought it would be (\textless{}50\%) repeat this section.

\end{enumerate}


\section{Measurement Setup: Measuring R1/T1 Dispersion Profile}
\label{\detokenize{stelar-ffc:measurement-setup-measuring-r1-t1-dispersion-profile}}\begin{enumerate}
\sphinxsetlistlabels{\arabic}{enumi}{enumii}{}{.}%
\item {} 
\sphinxAtStartPar
Navigate to Actions =\textgreater{} Profile Wizard.

\item {} 
\sphinxAtStartPar
Click Synchronise Parameters at the bottom of the wizard to copy over the parameters that we set up previously.

\item {} 
\sphinxAtStartPar
Change the run parameters (frequency range, number of points, number of repetitions per point etc) on the right hand side of the wizard.

\item {} 
\sphinxAtStartPar
Set filename via Export File =\textgreater{} Change.

\end{enumerate}

\begin{sphinxadmonition}{note}{Note:}
\sphinxAtStartPar
The total run\sphinxhyphen{}time for an experiment will be (approximately) = Profile Points x Repetition Points x Tau Blocks x Delay Time.
\end{sphinxadmonition}
\begin{enumerate}
\sphinxsetlistlabels{\arabic}{enumi}{enumii}{}{.}%
\setcounter{enumi}{4}
\item {} 
\sphinxAtStartPar
Click execute in the wizard to start the run.

\item {} 
\sphinxAtStartPar
The software will execute the first experiment (at the first frequency) and tell you whether it recommends you continue or not. If you have gone through the setup steps correctly this should work…

\item {} 
\sphinxAtStartPar
When the experiment has finished a table of data with the Frequency, T1, R1 etc will be output as a .sef file.

\end{enumerate}


\section{Shut Down}
\label{\detokenize{stelar-ffc:shut-down}}\begin{enumerate}
\sphinxsetlistlabels{\arabic}{enumi}{enumii}{}{.}%
\item {} 
\sphinxAtStartPar
Turn off heater.

\item {} 
\sphinxAtStartPar
Turn off Magenet from software

\item {} 
\sphinxAtStartPar
Turn off TX.

\item {} 
\sphinxAtStartPar
Turn off Galden recirculator.

\item {} 
\sphinxAtStartPar
Turn off air.

\end{enumerate}


\chapter{Novocontrol Broadband Dielectric Spectrometer}
\label{\detokenize{novocontrol-dielectric:novocontrol-broadband-dielectric-spectrometer}}\label{\detokenize{novocontrol-dielectric::doc}}

\section{Setting up a measurement in software}
\label{\detokenize{novocontrol-dielectric:setting-up-a-measurement-in-software}}\begin{enumerate}
\sphinxsetlistlabels{\arabic}{enumi}{enumii}{}{.}%
\item {} 
\sphinxAtStartPar
Start \sphinxstylestrong{WinDeta} (if it isn’t already) and navigate to \sphinxstylestrong{Temp. Controller}.

\item {} 
\sphinxAtStartPar
Select \sphinxstylestrong{Initialize from Controller}. If this has worked correctly then you should see the temperature of the sample, gas stream and Dewar in the \sphinxstylestrong{Status} window.

\end{enumerate}

\begin{sphinxadmonition}{note}{Note:}
\sphinxAtStartPar
You can choose whatever window placement you like within \sphinxstylestrong{WinDeta} but I’ve found that the most efficient is to tile the \sphinxstylestrong{Status}, \sphinxstylestrong{Message}, \sphinxstylestrong{Temperature Log} and \sphinxstylestrong{Online} windows together.
\end{sphinxadmonition}
\begin{enumerate}
\sphinxsetlistlabels{\arabic}{enumi}{enumii}{}{.}%
\setcounter{enumi}{2}
\item {} 
\sphinxAtStartPar
Navigate to \sphinxstylestrong{File =\textgreater{} Set File Names} (important to change the name of the output file immediately so that you do not overrite someone else’s measurement. )

\end{enumerate}

\sphinxAtStartPar
4. Navigate to \sphinxstylestrong{Measurement =\textgreater{} Sample Specification} and enter the required information. The \sphinxstylestrong{Description} box should contain information about your
sample (this information will appear at the top of any data you produce). The \sphinxstylestrong{Sample diameter} and \sphinxstylestrong{Sample thickness} (in mm) should also be entered \sphinxhyphen{} figure 4 shows how these values should be defined. For standard measurements (i.e. sample between two round electrodes), the \sphinxstylestrong{Cell Stray} should be set to \sphinxstylestrong{1} and \sphinxstylestrong{Spacer Area} can be set to 0. The checkbox for \sphinxstylestrong{Use Dielectric Sample Cell} should be checked.

\sphinxAtStartPar
This set of documentation contains standard operating procedures and guides for using SMP equipment, laboratories and proceses therein.
The documentation is \sphinxstyleemphasis{only} meant to suppliment proper training: you should always be trained by a competent user before attempting to use a piece of equipment.



\renewcommand{\indexname}{Index}
\printindex
\end{document}